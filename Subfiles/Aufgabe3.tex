\documentclass[crop=false]{standalone}
\usepackage[utf8]{inputenc}
\usepackage{amsmath}
\usepackage[dvipsnames]{xcolor}
\usepackage{pdfpages}
\usepackage{enumerate}
\usepackage{amssymb}
\usepackage[framemethod=default]{mdframed}
\usepackage[nomarginpar,left=2cm,right=2cm,top = 2cm, bottom = 2cm]{geometry}

\renewcommand{\thesubsection}{\thesection.\alph{subsection}}
\renewcommand{\thesubsubsection}{\thesection.\alph{subsection}.\roman{subsubsection}}

\mdfdefinestyle{theoremstyle}{%
linecolor=black,linewidth=.3pt,%
frametitlerule=true,%
frametitlebackgroundcolor=blue!5,
innertopmargin=\topskip,nobreak=true,
}

\mdfdefinestyle{style2}{frametitle={},%
             linewidth=.3pt,topline=true,backgroundcolor=blue!3!green!8!}

\mdtheorem[style=theoremstyle]{task}{Angabe}

\newmdenv[style = style2,title=false]{solution}

\begin{document}\begin{task}[Erreichbarkeit und Zustandsregler]
Betrachten Sie das LTI-System

$$ 
\begin{aligned} \dot{\mathbf{x}} &=\mathbf{A x}+\mathbf{b} u, \quad \mathbf{x}(0)=\mathbf{x}_{0} \\ y &=\mathbf{c}^{T} \mathbf{x} \end{aligned}
 $$
mit den Matrizen
\[
\mathbf{A}=\left[\begin{array}{ccc}{1} & {0} & {0} \\ {0} & {-1} & {0} \\ {0} & {-1} & {1}\end{array}\right], \quad \mathbf{b}=\left[\begin{array}{c}{\alpha} \\ {\beta} \\ {0}\end{array}\right], \quad \mathbf{c}^{T}=\left[\begin{array}{lll}{1} & {0} & {0}\end{array}\right]
\]
und den Parametern $\alpha, \beta \in \mathbb{R}$
 \begin{enumerate}[i]
  \item Geben Sie für die Parameterwerte $\alpha=2, \beta=1$ den erreichbaren Unterraum $\mathcal{R}$ an.
\begin{solution}
\[ \mathcal{R} = \text{span}\left\{ \mathbf{b}, \mathbf{Ab}, \mathbf{A^2 b}, \right\} =
\text{span}\left\{
\begin{pmatrix}
2 \\ 1 \\ 0
\end{pmatrix},
\begin{pmatrix}
2 \\ -1 \\ -1
\end{pmatrix},
\begin{pmatrix}
2 \\ 1 \\ 0
\end{pmatrix}
 \right\}
 =\text{span}\left\{
\begin{pmatrix}
2 \\ 1 \\ 0
\end{pmatrix},
\begin{pmatrix}
2 \\ -1 \\ -1
\end{pmatrix}
 \right\}
 \]
\end{solution}
  \item Gibt es für die Parameterwerte $\alpha=2, \beta=1$ und $\mathbf{x}_{0}=0$ eine Steuerung $u(t) \in C^{0}([0, T])$ mit $T<\infty$ so, dass $\mathbf{x}(T)=[2,0,1]^{T}$ gilt? Begründen Sie Ihre Aussage!
\begin{solution}
Es ist zu überprüfen ob $\mathbf{x}(T)=[2,0,1]^{T} \in \mathcal{R}$ ist:

\[\text{det}
\begin{pmatrix}
2 & 2 & 2\\
1 & -1 & 0\\
0 & -1 & 1
\end{pmatrix}
=
-2 -2 -(2) = -2
\]
Der Vektor $\mathbf{x}(T)=[2,0,1]^{T}$ ist linear unabhängig von den Basisvektoren von $\mathcal{R}$ und liegt somit nicht im Vektorraum der erreichbaren Zustände. Dieser Zustand kann durch keine Steuerung erreicht werden.
\end{solution}
  \item Bestimmen Sie falls möglich jenen Wertebereich der Parameter $\alpha, \beta \in \mathbb{R},$ für den $\operatorname{dim}(\mathcal{R})=1$ gilt. Begründung!
\begin{solution}
Dies ist erreicht, wenn $\mathbf{b}$ ein rechtseigenvektor von $\mathbf{A}$ ist.

Eigenwerte von $\mathbf{A}$: $
\text{det}
\begin{pmatrix}
1-\lambda & 0 & 0\\
0 & -1-\lambda & 0\\
0 & -1 & 1-\lambda
\end{pmatrix}
=\left(1-\lambda\right)^2\left(-1-\lambda\right)=0, \quad \lambda_{1,2} = 1, \lambda_3 = -1$


Eigenvektor zu $\lambda = 1$:
\[ \text{Ker}\begin{pmatrix}
0 & 0 & 0 \\
0 & -2 & 0\\
0 & -1 & 0
\end{pmatrix} = \text{span}\left\{ \begin{pmatrix}
1 \\ 0 \\ 0
\end{pmatrix}, \begin{pmatrix}
0 \\ 0 \\ 1
\end{pmatrix}\right\} \]
Für ein beliebiges $\alpha$ und $\beta=0$ wird $\mathbf{b}$ also zum rechtseigenvektor von $\mathbf{A}$ und der erreichbare Unterraum $\mathcal{R}$ wird ein eindimesnionaler Vektorraum.
\end{solution}
\end{enumerate}
\end{task}
\end{document}