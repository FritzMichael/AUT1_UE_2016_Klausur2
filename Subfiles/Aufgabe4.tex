\documentclass[crop=false]{standalone}
\usepackage[utf8]{inputenc}
\usepackage{amsmath}
\usepackage[dvipsnames]{xcolor}
\usepackage{pdfpages}
\usepackage{enumerate}
\usepackage{amssymb}
\usepackage[framemethod=default]{mdframed}
\usepackage[nomarginpar,left=2cm,right=2cm,top = 2cm, bottom = 2cm]{geometry}

\renewcommand{\thesubsection}{\thesection.\alph{subsection}}
\renewcommand{\thesubsubsection}{\thesection.\alph{subsection}.\roman{subsubsection}}

\mdfdefinestyle{theoremstyle}{%
linecolor=black,linewidth=.3pt,%
frametitlerule=true,%
frametitlebackgroundcolor=blue!5,
innertopmargin=\topskip,nobreak=true,
}

\mdfdefinestyle{style2}{frametitle={},%
             linewidth=.3pt,topline=true,backgroundcolor=blue!3!green!8!}

\mdtheorem[style=theoremstyle]{task}{Angabe}

\newmdenv[style = style2,title=false]{solution}

\begin{document}
\begin{task}
Gegeben ist das LTI-System
\[
\dot{\mathbf{x}}=\left[\begin{array}{cc}{0} & {6} \\ {3} & {-3}\end{array}\right] \mathbf{x}+\left[\begin{array}{l}{2} \\ {1}\end{array}\right] u
\]
 \begin{enumerate}[i]
     \item Zeigen Sie, dass für die Dimension des erreichbaren Unterraums $\operatorname{dim}(\mathcal{R})=1$ gilt.
     \begin{solution}
     \[\mathcal{R} = \text{span}\left\{\mathbf{b}, \mathbf{Ab} \right\}
     =
     \text{span}\left\{
     \begin{pmatrix}
     2\\1
     \end{pmatrix},
     \begin{pmatrix}
     6\\3
     \end{pmatrix}
     \right\}
     =
     \text{span}\left\{
     \begin{pmatrix}
     2\\1
     \end{pmatrix}
     \right\}
     \]
     $\mathbf{b}$ und $\mathbf{Ab}$ sind linear abhängig, $\mathcal{}$ wird also nur von einem Vektor aufgespannt und besitzt damit die Dimension 1.
     \end{solution}
     \item Transformieren Sie das System in ein erreichbares Teilsystem und ein Restsystem.
Geben Sie die Transformationsvorschrift sowie das System in transformierten
Koordinaten $\mathbf{z}$ explizit an, und kennzeichnen Sie das erreichbare Teilsystem.
\begin{solution}
 Transformationsvorschrift: $\mathbf{x}=\mathbf{V} \mathbf{z}, \ \mathbf{z}=\mathbf{V}^{-1} \mathbf{x}$
 
 In der Transformationsmatrix ist der Basisvektor des erreichbaren Unterraums und ein Komplementärvektor enthalten:
 
 \[\mathbf{V} = 
 \begin{pmatrix}
 2 & 1 \\
 1 & 0
 \end{pmatrix}, \quad 
 \mathbf{V}^{-1} = 
 \begin{pmatrix}
 0 & 1 \\
 1 & -2
 \end{pmatrix},
 \]
 
 \[\dot{\mathbf{z}} = \mathbf{V}^{-1} \dot{\mathbf{x}} = \mathbf{V}^{-1} \mathbf{A} \mathbf{V} \mathbf{z} + \mathbf{V}^{-1} \mathbf{b} u
 \]
 
 In transformierten Koordinaten ergibt sich das System dann zu:
 
 \[\dot{\mathbf{z}} = \begin{pmatrix}3&3\\0&-6\end{pmatrix} \mathbf{z} + \begin{pmatrix}1\\0\end{pmatrix} u
 \]
 
 Das erreichbare Teilsystem befindet sich in der ersten Zeile und lässt sich durch den Eingang $u$ beeinflussen. Das nicht-erreichbare Teilsystem befindet sich in der zweiten Zeile und lässt sich weder indirekt durch den ersten Zustand, noch direkt durch den Eingang $u$ beeinflussen (autonome DGL).
\end{solution}
 \end{enumerate}
\end{task}
\end{document}