\documentclass[crop=false]{standalone}
\usepackage[utf8]{inputenc}
\usepackage{amsmath}
\usepackage[dvipsnames]{xcolor}
\usepackage{pdfpages}
\usepackage{enumerate}
\usepackage{amssymb}
\usepackage[framemethod=default]{mdframed}
\usepackage[nomarginpar,left=2cm,right=2cm,top = 2cm, bottom = 2cm]{geometry}

\renewcommand{\thesubsection}{\thesection.\alph{subsection}}
\renewcommand{\thesubsubsection}{\thesection.\alph{subsection}.\roman{subsubsection}}

\mdfdefinestyle{theoremstyle}{%
linecolor=black,linewidth=.3pt,%
frametitlerule=true,%
frametitlebackgroundcolor=blue!5,
innertopmargin=\topskip,nobreak=true,
}

\mdfdefinestyle{style2}{frametitle={},%
             linewidth=.3pt,topline=true,backgroundcolor=blue!3!green!8!}

\mdtheorem[style=theoremstyle]{task}{Angabe}

\newmdenv[style = style2,title=false]{solution}

\begin{document}
\begin{task}[Beobachtbarkeit und Beobachterentwurf]
Betrachten Sie das LTI-System
\[
\begin{aligned} \dot{\mathbf{x}} &=\left[\begin{array}{ccc}{0} & {1} & {0} \\ {0} & {0} & {1} \\ {1} & {2} & {-1}\end{array}\right] \mathbf{x}+\left[\begin{array}{l}{0} \\ {0} \\ {1}\end{array}\right] u, \quad \mathbf{x}(0)=\mathbf{x}_{0} \\ y &=\left[\begin{array}{lll}{1} & {0} & {1}\end{array}\right] \mathbf{x}+u \end{aligned}
\]
\begin{enumerate}[i]
    \item Berechnen Sie einen Zustandsregler $u=\mathbf{k}^{T} \mathbf{x}$ so, die Eigenwerte des geschlossenen Kreises bei $\{-1,-2,-2\}$ platziert werden.
\begin{solution}
Das gewünschte charakteristische Polynom lautet:
\[ p(\lambda) = (\lambda +1) (\lambda +2)^2 = \lambda^3 + 5 \lambda^2 + 8 \lambda + 4 \]
Die Koeffizienten dieses Polynoms lauten: $\alpha_0 = 4, \alpha_1=8, \alpha_2 = 5$

\[\mathbf{e}^T_3 = \mathbf{t}_1^T \mathbf{M}_R\ = \mathbf{t}_1^T \begin{pmatrix}
0 & 0 & 1\\
0 & 1 & -1\\
1 & -1& 3
\end{pmatrix} \rightarrow \mathbf{t}_1^T = \begin{pmatrix}
1&0 &0
\end{pmatrix}\]
Einsetzen in die Formel nach Ackermann:
\[
\mathbf{k}^T = -\alpha_0 \mathbf{t}_1^T -\alpha_1 \mathbf{t}_1^T \mathbf{A} -\alpha_2 \mathbf{t}_1^T \mathbf{A}^2 - \mathbf{t}_1^T \mathbf{A}^3 = \begin{pmatrix}
-5&-10 &-4
\end{pmatrix}
\]
Ergebnis wurde durch Berechnen der Eigenwerte von $\mathbf{A}+\mathbf{k}^T \mathbf{b}$ bestätigt.
\end{solution}
\item Nun sei ein Zustandsregler $u=-x_{1}-5 x_{2}-2 x_{3}$ gegeben, und der Anfangswert
des Systems sei $\mathbf{x}_{0}=[2,1,1]^{T}$.
\begin{enumerate}[A]
    \item Zeigen Sie, dass der Zustandsregler die Ruhelage $\mathbf{x}_{s}=0$ des geschlossenen
Kreises zwar stabilisiert aber nicht asymptotisch stabilisiert.
\begin{solution}
Dynamikmatrix des geschlossenen Kreises:
\[\overline{\mathbf{A}} = \mathbf{A} + \mathbf{b \mathbf{k}^T} = 
\begin{pmatrix}
0&1&0\\
0&0&1\\
1&2&-1
\end{pmatrix}+
\begin{pmatrix}
0\\
0\\
1
\end{pmatrix}
\begin{pmatrix}
-1 & -5 & -2
\end{pmatrix}
=
\begin{pmatrix}
0&1&0\\
0&0&1\\
0&-3&-3
\end{pmatrix}
\]
Eigenwerte der Matrix $\overline{\mathbf{A}}$ um die Stabilität beurteilen zu können:
\[\text{det}
\begin{pmatrix}
-\lambda&1&0\\
0&-\lambda&1\\
0&-3&-3-\lambda
\end{pmatrix}=\lambda^2 \left( -3 -\lambda \right) - 3\lambda =0, \quad \lambda_1 = 0\]
\[\lambda^2 + 3\lambda +3 = 0 \rightarrow \lambda_{2,3} = \frac{-3 \pm \sqrt{9 - 12}}{2} = \frac{-3 \pm i\sqrt{3}}{2} \]

Die Realteile aller Eigenwerte sind $\leq 0$ aber nicht $<0$, somit ist die Ruhelage stabil, aber nicht asymptotisch stabil.
\end{solution}
\item Berechnen Sie den Wert der Ausgangsgröße $y(t)$ des geschlossenen Kreises
für den Zeitpunkt $t=0$.
\begin{solution}
Die Ausgangsgröße $y(t)$ hängt nur vom Zustandsvektor $\mathbf{x}$ zum Zeitpunkt t ab. $\mathbf{x}(t=0)$ ist bekannt, somit ergibt sich $y(t=0)$:

\[ y(t=0) = \mathbf{c}^T \mathbf{x}(t=0) = \begin{pmatrix}
1 & 0 & 1
\end{pmatrix} \begin{pmatrix}
2 \\ 1 \\ 1
\end{pmatrix} = 3\]
\end{solution}
\end{enumerate}
\end{enumerate}
\end{task}
\end{document}