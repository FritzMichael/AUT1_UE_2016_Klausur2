\documentclass[crop=false]{standalone}
\usepackage[utf8]{inputenc}
\usepackage{amsmath}
\usepackage[dvipsnames]{xcolor}
\usepackage{pdfpages}
\usepackage{enumerate}
\usepackage{amssymb}
\usepackage[framemethod=default]{mdframed}
\usepackage[nomarginpar,left=2cm,right=2cm,top = 2cm, bottom = 2cm]{geometry}

\renewcommand{\thesubsection}{\thesection.\alph{subsection}}
\renewcommand{\thesubsubsection}{\thesection.\alph{subsection}.\roman{subsubsection}}

\mdfdefinestyle{theoremstyle}{%
linecolor=black,linewidth=.3pt,%
frametitlerule=true,%
frametitlebackgroundcolor=blue!5,
innertopmargin=\topskip,nobreak=true,
}

\mdfdefinestyle{style2}{frametitle={},%
             linewidth=.3pt,topline=true,backgroundcolor=blue!3!green!8!}

\mdtheorem[style=theoremstyle]{task}{Angabe}

\newmdenv[style = style2,title=false]{solution}

\begin{document}
\begin{task}
Gegeben ist das LTI-System
\[
\begin{aligned} \dot{\mathbf{x}} &=\left[\begin{array}{cc}{-1} & {-4} \\ {1} & {3}\end{array}\right] \mathbf{x}+\left[\begin{array}{c}{3} \\ {-1}\end{array}\right] u \\ y &=\left[\begin{array}{cc}{0} & {2}\end{array}\right] \mathbf{x}-3 u \end{aligned}
\]
 \begin{enumerate}[i]
     \item Zeigen Sie, dass das System vollständig beobachtbar ist, und entwerfen Sie einen
vollständigen Beobachter so, dass die Eigenwerte des Fehlersystems bei $\{-2,-3\}$
liegen.
     \begin{solution}
     Prüfen ob das System vollständig beobachtbar ist, ob $\mathcal{O}$ den ganzen $\mathbb{R}^2$ aufspannt:
     \[\mathcal{O} = \text{span}\left\{
     \begin{pmatrix}0 & 2\end{pmatrix},\begin{pmatrix}2 & 6\end{pmatrix}\right\} = \mathbb{R}^2 \]
     Berechnen der Koeffizienten des gewünschten charakteristischen Polynoms:
     \[p(s) = \left( s+3 \right) \left( s+2 \right) = s^2 + \alpha_1 s + \alpha_0 =s^2 + 5s + 6 \rightarrow \alpha_0 = 6, \ \alpha_1 = 5  \]
     Berechnen des Vektors $\hat{\mathbf{t}}_1$:
     \[
     \mathbf{e}_2 = \mathbf{M}_O \hat{\mathbf{t}}_1 =
     \begin{pmatrix}0 \\ 1\\\end{pmatrix} = 
     \begin{pmatrix}0 & 2\\ 2 & 6\end{pmatrix}\hat{\mathbf{t}}_1
     \rightarrow \hat{\mathbf{t}}_1 = \begin{pmatrix}\frac{1}{2} \\ 0\end{pmatrix}
     \]
     
     Berechnung von $\hat{\mathbf{k}}$:
     
     \[\hat{\mathbf{k}} = - \alpha_0 \hat{\mathbf{t}}_1 - \alpha_1 \mathbf{A} \hat{\mathbf{t}}_1 - \mathbf{A}^2 \hat{\mathbf{t}}_1 = \begin{pmatrix} 1 \\ -\frac{7}{2}\end{pmatrix}\]
     
     Damit kann der vollständige Beobachter berechnet werden:
     \[
\begin{aligned} \dot{\mathbf{z}} &=\left(\mathbf{A}+\hat{\mathbf{k}} \mathbf{c}^{T}\right) \mathbf{z}+(\mathbf{b}+\hat{\mathbf{k}} d) u-\hat{\mathbf{k}} y \\ \hat{\mathbf{x}} &=\mathbf{E} \mathbf{z} \end{aligned} \quad \rightarrow 
\begin{aligned} \dot{\mathbf{z}} &= 
\begin{pmatrix}
-1 & -2 \\ 1 & -4
\end{pmatrix} \mathbf{z}+
\begin{pmatrix}
0 \\ \frac{19}{2}
\end{pmatrix}
u-
\begin{pmatrix}
1 \\ -\frac{7}{2}
\end{pmatrix}
y \\ \hat{\mathbf{x}} &=\mathbf{E} \mathbf{z} \end{aligned}
\]
     \end{solution}
     \item Welche Gröben des Systems müssen Sie kennen, um den Beobachter aus Punkt
(i) einsetzen zu können? Begründung!
 \begin{solution}
 Der Eingang $u$ und der Ausgang $y$ der Strecke sind Eingänge des Beobachters, müssen also bekannt sein. Zur Berechnung des Beobachters müssen $\mathbf{A}, \mathbf{b}, \mathbf{c}^T$ und $d$ bekannt sein. Also muss jede Größe des Beobachters bekannt sein, außer die Zustände $\mathbf{x}$ der Strecke.
 \end{solution}
 \end{enumerate}
\end{task}
\end{document}